% \iffalse meta-comment
%  This file should be thesis.dtx
%
%%%%%%%%%%%%%%%%%%%%%%%%%%%%%%%%%%%%%%%%%%%%%%%%%%%%%%%%%%%%%%%%%%
%%                                                               %
%%    Thesis and Thesis Document Class for LaTeX2e               %
%%    South Dakota School of Mines and Technology                %
%%    By Larry Pyeatt                                            %
%%    Updated January 2017                                       %
%%    Written in May 2013                                        %
%%                                                               %
%%    Based on the Texas Tech University                         %
%%    Thesis and Thesis Document Class for LaTeX2e               %
%%    by Larry Pyeatt       May 2010                             %
%%                                                               %
%%    Based on the LaTeX2.09 style for Colorado State University %
%%    created by                                                 %
%%     Thad Mauney     Fall l984                                 %
%%     Revised Summer 1985 - Scott Douglas                       %
%%     Greatly revised June 1985 - Gary Herron                   %
%%     Customized for me, by me Jun'85 - P. Fitzhorn             %
%%     Customized more and better Dec'85 - Gary Herron, Tom Wood %
%%     Re-written for LaTeX2e and fixed Aug 1997- Larry Pyeatt   %
%%     Modified May, 2017 by Larry Pyeatt:                       %
%%       1. "Table of Contents" entry added to Table of          %
%%          Contents.                                            %
%%       2. Completely re-wrote the way appendices are handled.  %
%%       3. Updated documentation.                               %
%%       4. Changed default bibliography style to ieeetr.        %
%%                                                               %
%%                                                               %
%%%%%    NOT GUARANTEED TO PASS GRADUATE SCHOOL STANDARDS    %%%%%
%%        BUT CLOSE ENOUGH NOT TO BE A WASTE OF TIME             %
%%%%%%%%%%%%%%%%%%%%%%%%%%%%%%%%%%%%%%%%%%%%%%%%%%%%%%%%%%%%%%%%%%
%
%  This file is distributed in doctex format and contains both
%  documentation and macros in one file.  You need to generate
%  the documentation and the class file as follows:
%
%  STEP 1 - save the dtx file
%    Save this file as ``thesis.dtx''
%  STEP 2 - generate documentation
%    Type ``latex thesis.dtx'' to create thesis.dvi
%    You can print thesis.dvi or view it to see how
%    the thesis class works.
%  STEP 3 - generate the class file
%    1. Create a new file ``thesis.drv'' that contains:
%
%       \input docstrip
%       \generateFile{thesis.cls}{t}{\from{thesis.dtx}{class}}
%       \end
%       
%       If you cut and paste from here, don't forget to delete the `%' 
%       characters at the beginning of each line.
%
%    2. Type ``latex thesis.drv'' to  create the .cls file. 
%
%%%%%%%%%%%%%%%%%%%%%%%%%%%%%%%%%%%%%%%%%%%%%%%%%%%%%%%%%%%%%%%%%%
% \fi
% \iffalse
%%
%% File `thesis.dtx'.
%% Copyright (C) 2017 by Larry Pyeatt
% \fi
% \iffalse
%<*driver>
\documentclass{ltxdoc}
\textwidth 5.5in
\textheight 8.5in
\evensidemargin 1in
\oddsidemargin 1in
\headsep 0in
\topmargin 0in
\headheight 0in
\parindent 0in
\usepackage{xspace}
\begin{document}
\OnlyDescription
\DocInput{thesis.dtx}
%\input docstrip
%\generateFile{thesis.cls}{t}{\from{thesis.dtx}{class}}
%\end
\end{document}
%</driver>
% \fi
% \iffalse
%<class>\NeedsTeXFormat{LaTeX2e}
%<class>\ProvidesClass{thesis}[2013/05/20 South Dakota School of Mines and Technology Thesis and Dissertation Class]
%<class>\DeclareOption{twocolumn}{\OptionNotUsed}
%<class>\DeclareOption{titlepage}{\OptionNotUsed}
%<class>\DeclareOption*{\PassOptionsToClass{\CurrentOption}{report}}
%<class>\ProcessOptions\relax
%<class>\LoadClass{report}
%<class>\RequirePackage{ifthen}
%<class>\RequirePackage[pagestyles]{titlesec}
%<class>\RequirePackage{caption}
%%<class>\DeclareCaptionFont{xipt}{\fontsize{11}{13}\mdseries}
%%<class>\RequirePackage[font=xipt,labelfont=bf]{caption}
%%<class>\DeclareCaptionFont{xpt}{\fontsize{10}{12}\mdseries}
%%<class>\RequirePackage[font=xpt,labelfont=bf]{caption}
%<class>\DeclareCaptionFont{xvpt}{\fontsize{10.5}{12.5}\mdseries}
%<class>\RequirePackage[font=xvpt,labelfont=bf]{caption}
%<class>\RequirePackage{enumitem}
\newcommand{\schoolname}{South Dakota School of Mines and Technology\\Rapid City, South Dakota}
\newif\ifthesiscitations
\thesiscitationsfalse
\flushbottom
\newcommand{\drheading}{
\renewpagestyle{plain}{\sethead[\usepage][][]{}{DRAFT \today}{\usepage}}
}
% \fi
%
%
%\title{Disseration and Thesis Document Class}
%\author{Larry D. Pyeatt}
%\maketitle
% \section{Document Class}
% Your main file should have |\documentclass[12pt]{thesis}| as the
% first line.  This document class is based on the |report| class, so
% any option available in the |report| class should be available, with
% two exceptions: the |twocolumn| and |titlepage| options are not
% available in the thesis class.  For example
% |\documentclass[11pt,twocolumn]{thesis}| will not work.
% You should be
% able to use any package that works for the report document class.
% For example, |\usepackage{graphicx}| should work.
% \vspace{\baselineskip}
%
% The thesis class also includes an optional bibliography style and
% set of macros for doing citations similar to the way they are
% specified in the Chicago manual of style.  The bibliography style
% and citation package are based on the standard \LaTeX\xspace chicago
% bibliograhpy style and package.  \DescribeMacro{\bibliographystyle}
% You can use |\bibliographystyle{thesis}| if you have the
% |thesis.bst| file.  In that case, you should also use
% |\usepackage{thesiscitations}| to get the matching citation macros.
% There is a separate PDF file describing those extended |\cite|
% macros.
% \vspace{\baselineskip}
%

% You may prefer to use a different bibliography style such as
% |\bibliographystyle{acm}| or |\bibliographystyle{ieeetr}|, and
% \emph{not} have the |\usepackage{thesiscitations}| command.  It is
% not compatible with those bibliography styles.  However, the extra
% macros in |thesiscitations| (|\citeN|, |\citeNP|, etc) are quite
% nice.  I would recommend using them unless you advisor objects. Read
% the documentation in |thesiscitations.pdf| for the list of all
% variations of the |\cite| macro.

%
% \section{Chapter Title Position}
%
% After the |\documentclass{}| statement, you can use one of these
% three commands to change the position of the chapter titles.  Chapter
% titles will be left justified unless you specify otherwise.
%
% \DescribeMacro{\lefttitles} The  |\lefttitles|
% command sets chapter titles to be left justified.
% \iffalse
\newcommand{\lefttitles}{
\def\ch@just{\raggedright}
}
\lefttitles
% \fi
%
% \DescribeMacro{\centertitles} The  |\centertitles|
% command sets chapter titles to be centered.
% \iffalse
\newcommand{\centertitles}{
\def\ch@just{\centering}
}
% \fi
%
% \DescribeMacro{\righttitles} The  |\righttitles|
% command sets chapter titles to be right justified.
% \iffalse
\newcommand{\righttitles}{
  \def\ch@just{\raggedleft}
}
% \fi
%
% \section{Optional Draft Commands}
% You can optionally specify the draft mode and spacing.  These
% commands allow you to print draft copies with single and 1.5 spacing
% to save paper. A doublespace draft mode is also available so you
% can check the final formatting.  If you do not
% specify a draft mode, then a final copy will be generated.
%
% \DescribeMacro{\ssdraft} shifts to single spacing and puts DRAFT
% in the heading of every page.
% \iffalse
\newcommand{\ssdraft}{
\setboolean{finalversion}{false}
\renewcommand{\doublespace}{\@normalsize\baselineskip\normalbaselineskip}
\drheading}
% \fi
%
% \DescribeMacro{\hsdraft} shifts to 1.5 spacing and puts DRAFT
% in the heading of every page.
% \iffalse
\newcommand{\hsdraft}{
\setboolean{finalversion}{false}
\renewcommand{\doublespace}{\@normalsize\baselineskip 1.45\normalbaselineskip}
\drheading}
% \fi
%
% \DescribeMacro{\dsdraft} shifts to double spacing and puts DRAFT
% in the heading of every page.  This mode should create a document
% that is identical to your final copy except for the DRAFT heading.
% \iffalse
\newcommand{\dsdraft}{
\setboolean{finalversion}{false}
\renewcommand{\doublespace}{\@normalsize\baselineskip 1.65\normalbaselineskip}
\drheading}
% \fi
%
% \section{Definitions}
%
% After the optional |\ssdraft|, |\hsdraft|, or |\dsdraft| statement, 
% you have to define some
% information that will be used for the cover sheet, running header, etc. 
% each of the following macros takes one argument.
%
% \DescribeMacro{\doctype}  defines the type of document and should
% be written as either
% |\doctype{thesis}| or |\doctype{dissertation}|.
% \iffalse
\newcommand{\doctype}[1]{\gdef\Zdoctype{#1}}
\doctype{---}
% \fi
%
% \DescribeMacro{\title} works just like in any other \LaTeX\xspace document.
% \iffalse
\renewcommand{\title}[1]{\gdef\Ztitle{#1}}
\title{---}
% \fi
%
% \DescribeMacro{\author} works just like in any other \LaTeX\xspace document.
% \iffalse
\renewcommand{\author}[1]{\gdef\Zauthor{#1}}
\author{---}
% \fi
%
% \DescribeMacro{\degree} defines the type of degree.  Here are some examples:\\
% |\degree{Doctor of Philosophy in Underwater Basket Weaving}|\\
% |\degree{Master of Science in Applied Handball}|\\
% |\degree{Master of Arts in Finger Painting}|\\ Use whatever is
% appropriate for your degree.
% \iffalse
\newcommand{\degree}[1]{\gdef\Zdegree{#1}}
\degree{---}
% \fi
%
% \DescribeMacro{\defensedate} defines the date of the oral defense.
% For example:\\
% |\defensedate{January 12, 1988}|
% \iffalse
\newcommand{\defensedate}[1]{\gdef\Zdefdate{#1}}
\defensedate{---}
%\fi
%
% \DescribeMacro{\gradyear} defines the year you are graduating, It should be 
% four digits.
% \iffalse
\newcommand{\gradyear}[1]{\gdef\Zyear{#1}}
\gradyear{---}
% \fi
%
% \DescribeMacro{\department} defines the department that is granting
% your degree, examples:\\
% |\department{Department of Mathematics and Computer Science}|\\
% |\department{Department of Zoology}|\\
% |\department{Department of Redundancy Department}|
% \iffalse
\newcommand{\department}[1]{\gdef\Zdepartment{#1}}
\department{---}
% \fi
%
% \DescribeMacro{\signatureline} adds a line to the signature page where someone needs to sign. Examples:\\
% |\signatureline{Major Professor --- Dimm Whitt, Ph.D., Department of Zoology}|\\
% |\signatureline{Grad. Div. Rep. --- E.\ Nigma, Ph.D., Department of Philosophy}|\\
% |\signatureline{Committee Member --- Chip Munk, Ph.D., Department of Zoology}|\\
% |\signatureline{Head of the Zoology Department --- Earl E.\ Bird, Ph.D.}|\\
% |\signatureline{Dean of Graduate Education --- Raney Daze}|\\
% \LaTeX does not care what you type as the argument, but the Graduate
% School does. Read their Thesis and Dissertation Writing Manual for further
% instructions on the signature lines.
% \iffalse
\newcommand{\make@signatureline}[1]{{%\footnotesize
\noindent\parbox{\textwidth}{\noindent\parbox[t][2\baselineskip]{4in}{\noindent\rule{4in}{1pt}\\\raggedright\noindent #1}\hfill\noindent\parbox[t]{1.5in}{\rule{1.5in}{1pt}\\\noindent Date}}}}
\newcommand{\make@signatures}{{}%
}
\newcommand{\signatureline}[1]{\g@addto@macro\make@signatures{\par\vfill\noindent\make@signatureline{#1}}}
% \fi
%
% \section{Sections within the document}
% The thesis or dissertation is divided into 3 main sections: preliminaries, body, 
% and supplementaries.  There are 3 commands used to switch from one
% main section to the next.
%
% \DescribeMacro{\preliminaries} The preliminaries section is for maketitle,
% table of contents, etc and comes directly after the 
% the |\begin{document}| command.  
% \iffalse
\newcommand{\preliminaries}{\doublespace\pagestyle{plain}\eject
  \pagenumbering{roman}\setcounter{page}{1}}
% \fi
%
% \DescribeMacro{\body} The body section is for the main body of your work and
% it should come directly after the last section in the preliminaries. 
% \iffalse
\newcommand{\body}{\doublespace\vfill\pagebreak\eject
  \pagenumbering{arabic}\setcounter{page}{1}\eject}
% \fi
%
% \DescribeMacro{\supplementaries} The supplementary section contains the  
% bibliography, appendices, glossary, index, and vita.  
% \iffalse
\newcommand{\supplementaries}{\renewcommand{\@chapapp}{Appendix}
\setcounter{chapter}{0}\renewcommand{\thechapter}{\Alph{chapter}}}
% \fi
%
%\section{Commands and Environments in the Preliminaries}
% There are several commands and environments for use within the 
% preliminaries section.  
%
%\DescribeMacro{\maketitle}  The |\maketitle|
% command creates the title page, just as it does for other \LaTeX\xspace classes.
% \iffalse
\renewcommand{\maketitle}{%
\null%
\singlespace%
\pagestyle{empty}%
\thispagestyle{empty}%
\vspace{-2\baselineskip}%
\begin{center}%
{{\fontsize{14pt}{18pt}\selectfont
\bf\parbox{\textwidth}{\begin{center}\Ztitle\end{center}}}}\\%
{%\large
%\vspace{0.75\baselineskip}%
by\\%
\Zauthor\\%
\vspace{0.75\baselineskip}%
A \Zdoctype\ submitted to the Graduate Division\\%
in partial fulfillment of the requirements for the degree of\\%
\vspace{0.75\baselineskip}%
\Zdegree\\%
\vspace{0.75\baselineskip}%
\schoolname\\%
\vspace{0.75\baselineskip}%
Date Defended: \Zdefdate\\%
}\end{center}%
%% \vspace{\baselineskip}%
%% \noindent\parbox{\textwidth}{Prepared by:}%
%% \vfill%
%% \make@signatureline{\Zauthor}%
%% \par%
%% \vspace*{\baselineskip}%
\noindent\parbox{\textwidth}{Approved by:}%
\vfill%
\vspace*{-\baselineskip}%
\make@signatures%
}
\renewcommand\section{\@startsection{section}{1}{\z@}%
{-2.5ex \@plus -1ex \@minus -.2ex}%
{0.8ex \@plus.2ex}%
{\normalfont\large\bfseries}}                                   
\renewcommand\subsection{\@startsection{subsection}{2}{\z@}%
{-2.25ex\@plus -1ex \@minus -.2ex}%
{0.8ex \@plus .2ex}%
{\normalfont\normalsize\bfseries}}
\renewcommand\subsubsection{\@startsection{subsubsection}{3}{\z@}%
{-2.25ex\@plus -1ex \@minus -.2ex}%
{0.8ex \@plus .2ex}%
{\normalfont\normalsize\bfseries}}
\renewcommand\paragraph{\@startsection{paragraph}{4}{\z@}%
{2.25ex \@plus 1ex \@minus .2ex}%
{-1em}%
{\normalfont\normalsize\bfseries}}
\renewcommand\subparagraph{\@startsection{subparagraph}{5}{\parindent}%
{2.25ex \@plus 1ex \@minus .2ex}%
{-1em}%
{\normalfont\normalsize\bfseries}}
% \fi
%
% \DescribeMacro{\makecopyright} The  |\makecopyright|
% command creates a copyright page. Read the section
% about copyright in the Thesis and Dissertation Writing Manual
% from the graduate school.
% \iffalse
\newcommand{\makecopyright}{\newpage\thispagestyle{empty}
\vspace*{0.3\textheight}%
\begin{center}%
  Copyright \copyright\ \Zyear, \Zauthor\\%
  All Rights Reserved%
\end{center}%
\vfill\vfill%
\newpage\doublespace}%
\def\@makeschapterhead#1{\newpage%
%\vspace*{-40\p@}%
{\parindent \z@ \ch@just%
\normalfont%
\interlinepenalty\@M%
\Large
\bfseries  #1\par\nobreak%
\vskip 40\p@
}}
\def\@makechapterhead#1{%
  {\parindent \z@ \ch@just \normalfont
  %{\centering\normalfont
    \ifnum \c@secnumdepth >\m@ne
    \Large
    \bfseries \@chapapp\space \thechapter
    \par\nobreak
    \vskip 20\p@
    \fi
    \interlinepenalty\@M
    \Large
    \bfseries #1\par\nobreak
    \vskip 40\p@
  }}
\def\@makeappendixhead#1{%
  {\parindent \z@ \ch@just \normalfont
  %%{\centering\normalfont
    \ifnum \c@secnumdepth >\m@ne
    \Large
    \bfseries Appendices
    \par\nobreak
    \vskip 20\p@
    \fi
    \interlinepenalty\@M
    \Large
    \bfseries #1\par\nobreak
    \vskip 40\p@
  }}
% \fi
%
% \DescribeEnv{abstract} The abstract environment is used
% the same as for any other \LaTeX\xspace document.  Just use
% |\begin{abstract} text of the abstract \end{abstract}| as usual.
% \iffalse
%%\renewenvironment{abstract}{\singlespace\newpage
\renewenvironment{abstract}{\newpage
%%  \addcontentsline{toc}{chapter}{\protect\numberline{}Abstract}
  \addcontentsline{toc}{chapter}{Abstract}
  \@makeschapterhead{\centerline{Abstract}}
  \vskip 20pt\par\singlespace}{\newpage}%
%% This version prints the thesis/dissertation title
%% \renewenvironment{abstract}{\singlespace
%%   \addcontentsline{toc}{chapter}{Abstract}
%%   \@makeschapterhead{\centerline{Abstract}}
%%   \vskip 20pt\par\noindent Title: \Ztitle\par
%%   \vskip 0.5\baselineskip\strut\\%
%%   \noindent}{\newpage}%
% \fi
%
% \DescribeEnv{acknowledgments} The acknowledgments environment 
% creates a new page with at title for acknowledgments.  It
% works very much like the |abstract| environment.
% \iffalse
%% \newenvironment{acknowledgments}{\singlespace\newpage
\newenvironment{acknowledgments}{\newpage
%%  \addcontentsline{toc}{chapter}{\protect\numberline{}Acknowledgments}
  \addcontentsline{toc}{chapter}{Acknowledgments}
  \@makeschapterhead{\centerline{Acknowledgments}}
  \vskip 20pt\par}{\newpage}
% \fi
%
% \DescribeMacro{\tableofcontents}  The |\tableofcontents|
% macro makes a new page with the table of contents on it.
% It works just like it does in other document classes
% \iffalse
\def\contentsname{Table of Contents}
\renewcommand{\tableofcontents}{\singlespace\newpage
  \addcontentsline{toc}{chapter}{Table of Contents}
  \@makeschapterhead{\centerline{Table of Contents}}
  \vskip 20pt\par\@starttoc{toc}}
% \fi
%
% \DescribeMacro{\listoftables} The |\listoftables| macro makes a new
% page with the list of tables on it.  It works just like it does in
% other document classes
% \iffalse
\renewcommand{\listoftables}{\singlespace\newpage
\addcontentsline{toc}{chapter}{List of Tables}
%%\addcontentsline{toc}{chapter}{\protect\numberline{}List of Tables}
  \@makeschapterhead{\centerline{List of Tables}}
  \vskip20pt
  \@starttoc{lot}}
% \fi
%
% \DescribeMacro{\listoffigures} The |\listoffigures| macro makes a
% new page with the list of figures on it.  It works just like it does
% in other document classes
% \iffalse
\renewcommand{\listoffigures}{\singlespace\newpage
  \addcontentsline{toc}{chapter}{List of Figures}
%%  \addcontentsline{toc}{chapter}{\protect\numberline{}List of Figures}
  \@makeschapterhead{\centerline{List of Figures}}
  \vskip20pt
  \@starttoc{lof}}
% \fi
%

% \DescribeEnv{genericlist} The |genericlist| environment and the
% |\gltitle| macro provide a way for you to include any additional
% lists at the beginning of your document, use them to create lists of
% acronynms, abbreviations, symbols, equations, keywords, etc.  You must build
% the table yourself within the environment, using the |table| or
% other environments.  Use |\gltitle| to set the title, then
% |\begin{genericlist}| text |\end{genericlist}| to create the list
% page(s).  Do this for each additional list that you want to include
% in your document.  The default title is ``Generic List''.  The
% |\gltitle| macro and the |genericlist| environment can be used
% multiple times to create as many lists as you need.
% \iffalse
\newenvironment{genericlist}{\singlespace\newpage
%%  \addcontentsline{toc}{chapter}{\protect\numberline{}List of Symbols and Acronyms}
  \addcontentsline{toc}{chapter}{\gl@title}
  \@makeschapterhead{\centerline{\gl@title}}
  \vskip 20pt\par}{\newpage}
% \fi

% \DescribeMacro{\gltitle} The |\gltitle| macro allows you to change
% the title of the following |genericlist| environment.  For example,
% |\gltitle{List of Acronyms and Abbreviations}| will set the title of
% the next generic list to ``List of Acronyms and Abbreviations''.
% \iffalse
\newcommand\gl@title{Generic List}
\newcommand\gltitle[1]{\renewcommand\gl@title{#1}}
% \fi
%

%
% \DescribeEnv{dedication} The dedication environment 
% creates a new page with title for the dedication.  It
% works very much like the |abstract| environment.
% \iffalse
%%\newenvironment{dedication}{\singlespace\newpage
\newenvironment{dedication}{\newpage
%%  \addcontentsline{toc}{chapter}{\protect\numberline{}Dedication}
  \addcontentsline{toc}{chapter}{Dedication}
  \@makeschapterhead{\centerline{Dedication}}
  \vskip 20pt\par}{\newpage}
% \fi
%
% \DescribeEnv{preface} The preface environment 
% creates a new page with at title for the preface.  It
% works very much like the |abstract| environment.
% \iffalse
%%\newenvironment{preface}{\singlespace\newpage
\newenvironment{preface}{\newpage
  %%  \addcontentsline{toc}{chapter}{\protect\numberline{}Preface}
  \addcontentsline{toc}{chapter}{Preface}
  \@makeschapterhead{\centerline{Preface}}
  \vskip 20pt\par}{\newpage}
% \fi
%
% \section{Commands and environments within the body}
% Commands and envionments within the body text should work as normal
% for the report document class.  Use |\chapter{}|, |\section{}|, etc.
% You may wish to write the chapter bodies in a separate files and just include 
% them at the appropriate place in the main \LaTeX\ file.
%
% \section{Commands and environments within the supplementaries}
% The supplementaries section contains the bibliography, appendices, 
% and other additional information that you may wish to include.
% There is a file named |thesis.bst| that is included in the SDSMT
% thesis package.
% If you do not have |thesis.bst|, or do not want to use it, then you can use
% the Chicago bibliography style (or any style appropriate for your field).
% Other than that, the bibliography should be created as you would
% for any other document class. 
%
% \iffalse
\renewcommand{\thebibliography}[1]{
 \@makeschapterhead{\centerline{Bibliography}}
 \vskip20pt
%% \addcontentsline{toc}{chapter}{\protect\numberline{}References}
 \addcontentsline{toc}{chapter}{Bibliography}
 \singlespace
 \list{[\arabic{enumi}]}{\settowidth\labelwidth{[#1]}
 \leftmargin\labelwidth 
 \advance\leftmargin\labelsep
   \advance\leftmargin\bibindent
   \itemindent -\bibindent
   \listparindent \itemindent
   \parsep \z@
 \usecounter{enumi}}
 \def\newblock{\hskip .11em plus .33em minus -.07em}
 \sloppy
 \sfcode `\.=1000\relax}
% \fi
%
% \DescribeEnv{appendices} The |appendices| environment is used to create
% the appendices.  Within this environment, the |\appendix| macro is used to
% create individual appendices.  The |\appendix| macro works just like the
% |\chapter| macro, but generates an appendix rather than a chapter.
% \iffalse
\newenvironment{appendices}{
\if@openright\cleardoublepage\else\clearpage\fi
 \addcontentsline{toc}{chapter}{Appendices}
 \thispagestyle{plain}\parindent\z@
 \parskip\z@ \@plus .3\p@\relax}
  {\clearpage}
% \fi
%
% \DescribeMacro{\appendix} The |\appendix| macro works like |\chapter|, but
% |\appendix| in used within the |appendix| environment.
% \iffalse
\renewcommand\appendix{\if@openright\cleardoublepage\else\clearpage\fi
                    \thispagestyle{plain}%
                    \global\@topnum\z@
                    \@afterindentfalse
                    \secdef\@appendix\@schapter}
\def\@appendix[#1]#2{\ifnum \c@secnumdepth >\m@ne
                         \refstepcounter{chapter}%
                         \typeout{\@chapapp\space\thechapter.}%
                         \addcontentsline{toc}{section}%
                         %%{\protect\numberline{Appendix \thechapter}}%
                         {Appendix \thechapter}%
                    \else
                    %% \addcontentsline{toc}{section}{#1}%
                    %% \addcontentsline{toc}{section}{}%
                    \fi
                    \chaptermark{#1}%
                    \addtocontents{lof}{\protect\addvspace{10\p@}}%
                    \addtocontents{lot}{\protect\addvspace{10\p@}}%
                    \if@twocolumn
                      \@topnewpage[\@makeappendixhead{#2}]%
                    \else
                      \@makeappendixhead{#2}%
                      \@afterheading
                    \fi}
% \fi
%
% \DescribeEnv{gloss} The |gloss| environment is used to create
% a glossary.  The environment starts a new page and puts an
% appropriate heading, you have to fill in the text yourself.
% Use |\begin{glossary}| |text| |\end{glossary}| to create a glossary.
% \iffalse
\newenvironment{gloss}{
 \newpage\@makeschapterhead{\centerline{Glossary of Terms}}
%% \addcontentsline{toc}{chapter}{\protect\numberline{}Glossary of Terms}
 \addcontentsline{toc}{chapter}{Glossary of Terms}
 \thispagestyle{plain}\parindent\z@
 \parskip\z@ \@plus .3\p@\relax}
 {\clearpage}
% \fi
%
% \DescribeEnv{abbreviations} The |abbreviations| environment is used to create
% list of abbreviations.  The environment starts a new page and puts an
% appropriate heading, you have to fill in the text yourself.
% Use |\begin{abbreviations}| |text| |\end{abbreviations}| to 
% create a list of abbreviations.
% \iffalse
\newenvironment{abbreviations}{
 \@makeschapterhead{\centerline{List of Abbreviations}}
 %% \addcontentsline{toc}{chapter}{\protect\numberline{}List of Abbreviations}
 \addcontentsline{toc}{chapter}{List of Abbreviations}
 \thispagestyle{plain}\parindent\z@
 \parskip\z@ \@plus .3\p@\relax}
 {\clearpage}
% \fi
%
% \DescribeMacro{\makeindex} The index generation macros work the same
% as in other document classes. Put |\usepackage{makeidx}| and
% |\makeindex| somewhere before
% |\begin{document}| if you want to use makeindex to create an index.
%
% \DescribeMacro{\index} Use the |\index| macro as described in
% ``The \LaTeX\xspace Companion'' and ``\LaTeX: a Document Preparation
%  Language'' to add entries to your index.
%
% \DescribeMacro{\printindex} The |\printindex| macro is used to insert
% an index created by the |makeindex| program.  
% The macro starts a new page and puts an
% appropriate heading, then inserts the index. 
% \iffalse
\def\indexname{Index}
\renewenvironment{theindex}{
% \newpage\singlespace
 \columnseprule \z@
 \columnsep 35\p@
 \twocolumn[\@makeschapterhead{\centerline{\indexname}}]%
%% \addcontentsline{toc}{chapter}{\protect\numberline{}Index}
 \addcontentsline{toc}{chapter}{Index}
 \thispagestyle{plain}\parindent\z@
 \parskip\z@ \@plus .3\p@\relax
 \let\item\@idxitem}
 {\onecolumn\clearpage}
% \fi
%
% \DescribeEnv{vita} The vita environment creates a new page with at
% title for the vita.  It works very much like the |abstract|
% environment. All SDSMT theses and dissertations must have a vita.
% Read the Thesis and Dissertation Writing Manual
% from the graduate school for specific instructions about the vita page. 
% \iffalse
\newenvironment{vita}{\newpage
%%  \addcontentsline{toc}{chapter}{\protect\numberline{}Vita}
  \addcontentsline{toc}{chapter}{Vita}
  \@makeschapterhead{\centerline{Vita}}
  \vskip 20pt\par}{\newpage}
% \fi
%
% \section{General use commands and defaults}
% \DescribeMacro{\singlespace} The |\singlespace| command is used to 
% switch to single spacing mode, and 
% \DescribeMacro{\doublespace} 
% |\doublespace|is used to switch to double spacing mode.
%NOTE: The doublespace command is overridden by the |\ssdraft| and |\hsdraft|
% commands!
% \iffalse
\newcommand{\singlespace}{\@normalsize \baselineskip \normalbaselineskip}
\newcommand{\doublespace}{\@normalsize \baselineskip 1.65\normalbaselineskip}
% \fi
%
% \iffalse
\widowpenalty=10000
\clubpenalty=10000
%% I don't know why the margins have to be set like this.  Makes no sense.
\textwidth 6.25in
\evensidemargin 0.375in
\oddsidemargin 0.375in
%% \evensidemargin 0.5in
%% \oddsidemargin 0.5in
%% \textwidth 6in
\textheight 9in
\footskip\baselineskip
%\addtolength{\textheight}{-\footskip}
\topmargin 0in
\headheight \normalbaselineskip
\headsep 0.25in
\widenhead{0.25in}{0.25in}
\renewpagestyle{plain}{\sethead[\usepage][][]{}{}{\usepage}}
\pagestyle{plain}
\addtolength{\topmargin}{-\headheight}
\addtolength{\topmargin}{-\headsep}
\setcounter{secnumdepth}{5}
\setcounter{tocdepth}{5}
\footnotesep 1.65\baselineskip
\setlength{\skip\footins}{1.65\baselineskip}
\renewcommand{\footnoterule}{\noindent\vspace*{-3pt}\rule{1.5in}{0.4pt}\vspace*{2.6pt}}
\newboolean{finalversion}
\setboolean{finalversion}{true}
\renewcommand\l@chapter[2]{%
  \ifnum \c@tocdepth >\m@ne
  \addpenalty{-\@highpenalty}%
  \vskip 1.0em \@plus\p@
  \setlength\@tempdima{1.5em}%
  \begingroup
  \parindent \z@ \rightskip \@pnumwidth
  \parfillskip -\@pnumwidth
  \leavevmode \bfseries
  \advance\leftskip\@tempdima
  \hskip -\leftskip
  #1\nobreak\normalfont\leaders\hbox{$\m@th
    \mkern \@dotsep mu\hbox{.}\mkern \@dotsep
    mu$}\hfill\nobreak\hb@xt@\@pnumwidth{\hss #2}\par
  \penalty\@highpenalty
  \endgroup
  \fi}
\renewcommand*\l@section{\@dottedtocline{1}{1.5em}{2.3em}}
\renewcommand*\l@subsection{\@dottedtocline{2}{3.8em}{3.2em}}
\renewcommand*\l@subsubsection{\@dottedtocline{3}{7.0em}{4.1em}}
\renewcommand*\l@paragraph{\@dottedtocline{4}{10em}{5em}}
\renewcommand*\l@subparagraph{\@dottedtocline{5}{12em}{6em}}
% \fi
%
% \section{Example}

% The SDSMT \LaTeX\ thesis package includes two files: |harvey.tex|
% and |harvey.bib| to provide examples of most of the commands
% provided by the package.  You can use them as a template for your
% thesis or dissertation.  Just re-name them or copy them to new
% files. and you are ready to start writing your thesis or
% dissertation!
